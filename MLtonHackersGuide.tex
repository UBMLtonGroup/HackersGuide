\documentclass{article}
\usepackage[margin=0.75in]{geometry}
\usepackage{hyperref}
\usepackage[utf8]{inputenc}
\usepackage{fancyvrb}
\usepackage{fancyhdr}
\usepackage{lastpage}
\usepackage{listings}
\usepackage{url}

\title{MLton Hacker's Guide (Do Panic)}
\author{Vasanth Subramanian, Bhushan Shitole, Nathan Burgers}
\date{}

\DefineShortVerb{\|}
\pagestyle{fancy}

\begin{document}
\maketitle
\tableofcontents
\clearpage

\section{Roadmap}
\subsection{Deliverables by the end of UB Fall Semester}
\begin{itemize}
\item The Hacker's Guide
  \begin{itemize}
  \item The directory structure of MLton
  \item How to edit the Garbage Collector
  \item The phases of compilation, to be stolen from current documentation
  \item Generating and Inspecting intermediate representations
  \item Running regression tests
  \item Debugging alterations
  \item Editing the build system
  \end{itemize}
\item MLton changes
  \begin{itemize}
  \item Option to enable / disable parallel (split) heap collection. This may be automatically detected depending on whether the program to be compiled uses parallel features
  \item Remove deprecated parts of the build system
  \end{itemize}
\end{itemize}
\subsection{Deliverables by the end of UB Spring Semester}
\begin{itemize}
  \item \textbf{TBD}
\end{itemize}

\section{Directory Structure}
\begin{description}
\item |basis-library/| \\
  This directory contains the version of the Standard-ML Basis Library that ships with MLton.

\item |benchmark/| \\
  Contains the performance benchmarks used for MLton's own performance   tests.

\item |bin/| \\
  A collection of shell scripts that aid in compilation

\item |doc/| \\
  Documentation about the internals of MLton. Contains the old Hacker's Guide and other miscellaneous documentation.

\item |mlton/| \\
  The source for the compiler proper. Most of the work you will be doing will take place inside this directory.
\begin{description}
\item |ast/| \\
  Algebraic Data Types representing the Abstract 
Syntax Tree that MLton creates after parsing the text of a Standard-ML program. This directory also contains files with meta-data on the various primitive data types, such as the size of the primitive |int| type provided in the Basis.
\item |atoms/| \\
  These are the atomic symbols used by MLton in many intermediate representations. 
\item |backend/| \\
  Lower-level intermediate representations and optimization passes. For instance, the RSSA and Machine intermediate representations, and the Register Allocation pass are located here.
\item |closure-convert/| \\
  Sources for the Closer Conversion optimization pass.
\item |cm/| \\
  The Parser for the deprecated Core-ML build system. This directory is kept for backwards-compatibility. Prefer to use ML-Build (|.mlb|) files to structure Standard-ML programs.
\item |codegen/| \\
  The lowest level of the compiler. This directory contains the passes that take code from the |Machine| intermediate representation to either: bytecode, x86 assembly, x86-64 assembly, or C. 
\item |control/| \\
  Contains systems to \emph{control} the compiler. Enables and runs various optimization passes.
\item |core-ml/| \\
  The Abstract Syntax Tree of the Core-ML language. This intermediate representation conforms to the specification for Core-ML in The Definition of Standard-ML. Core-ML is essentially Standard-ML without the module system.
\item |defunctorize/| \\
  Defunctorization is a compiler pass that converts all invocations of Functors into Structures. 
\item |elaborate/| \\
  \textbf{Nate: I do not fully understand this.}
\item |front-end/| \\
  Contains code to Parse and translate ML-Build (|.mlb|) files into a Compilation Unit.
\item |main/| \\
  Exactly what it sounds like, |main/| is the entry point for the   MLton compiler.
\item |match-compile/| \\
  Contains the intermediate pass that compiles Pattern Matching.
\item |ssa/| \\
  This directory contains the definition of the Static Single Assignment 1 and 2 intermediate representations, as well as the optimization passes that run on them. This is where the bulk of optimization takes place in MLton. Static Single Assignment is a typical canonical form for optimization in compilers, and a variant of it is used, for instance, as the basis for LLVM.
\item |xml/| \\
  \textbf{To be continued...}
\end{description}

\item |runtime/| \\
  

\end{description}

\section{Intermediate Representations}

\section{Backend}

\section{Runtime}

\subsection{Execution Model}

SML programs rely heavily on recursion. If each recursive call was
translated to a C function call, a stack overflow could occur
easily. To account for this common use case, MLton creates a stack in
the C heap and manage it in the runtime.

See the ``Execution Model'' section of Andrew Leibig's master's thesis
~\cite{Leibig13} for a good review of the execution model.

\subsection{Trampolining}

In MLton or MultiMLton, you may encounter a section of code in the
\texttt{MLton\_main()} function in \texttt{include/c-main.h} that looks
like the following code listing.

\lstset{language=C}
\begin{lstlisting}
        /* Trampoline */
        while (1) {
                cont=(*(struct cont(*)(void))cont.nextChunk)();
                /* (The above line is repeated 7 more times) */
        }      
\end{lstlisting}

Trampolining is used to implement the execution model described
above.  When a MLton-compiled program begins, after setting up the
runtime, the trampoline loop begins executing.  It is clear that this
loop will not cause the stack to grow, but it is not so clear how it
actually causes the program to execute.

\textbf{TODO: explain how the trampoline executes ML code}

See the ``Trampolining'' section of Andrew Leibig's master's thesis for more details ~\cite{Leibig13}.

\section{Foreign Function Interface}

Helpful documentation of FFI functions, such as \texttt{\_import} can
be found at the official MLton site's
\href{ForeignFunctionInterfaceSyntax page}{http://mlton.org/ForeignFunctionInterfaceSyntax}.

\section{MultiMLton}
\subsection{Threading}

The MultiMLton runtime creates a pthread\footnote{Pthreads are also
  known as ``OS-level threads'' or ``native threads.''} for each
processor on the system.  SML-level/green threads are multiplexed on
top of the pthreads (or ``virtual processors''). The reason for this
added complexity is that the runtime is capable of context switching
between green threads \textit{faster} than the OS is capable of
switching between pthreads.

\subsubsection{Implementation}

Generally, MLton/MultiMLton compiles programs by repeatedly
translating from intermediate language\footnote{The Intermediate
  Languages used by MLton are described in more detail at
  \url{http://mlton.org/IntermediateLanguage}} to intermediate
language, beginning with the abstract syntax tree language.  The
result of the final intermediate language, \texttt{Machine}, is passed
as input to the \texttt{Codegen} stage.  The \texttt{C-Codegen}
functor\footnote{C-Codegen is defined in
  \texttt{mlton/codegen/c-codegen/c-codegen.fun}.} defines
\texttt{declareMain}, which injects the definition of the C program's
\texttt{main} function.  This \texttt{main} function simply calls the
\texttt{MLton\_main} function.

To understand where this \texttt{MLton\_main} function comes from, we
must look at \texttt{include/c-main.h}. This header file
\texttt{\#define}s a macro called \texttt{MLtonMain} that expands to
define a number of functions. One of these functions is named
\texttt{MLton\_main}.

\textbf{TODO: expand on \texttt{MLton\_main}}

\textbf{TODO: summarize \texttt{basis-library/mlton/thread.sml}}

\bibliography{References}{}
\bibliographystyle{plain}

\end{document}
%%% Local Variables:
%%% mode: latex
%%% TeX-master: t
%%% End:
